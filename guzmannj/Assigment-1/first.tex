\documentclass[12pt]{article}
%\usepackage{times}
\usepackage{cite}
%this is a comment
\title{Zombie Apocalypse Prep}
\author{Jorge Guzman Nader: guzmannj, Christa Wright: wrighch3}





\begin{document}
\maketitle
\tableofcontents



\section{Problem}
Only 18\% of all Americans gets the weekly recommendations for cardiovascular and muscle-strengthening activity~\cite{Harvard}. 

This is alarming if we consider that poor diet and physical inactivity produce around 365,000 deaths per year (accounting for 15.2\% of all deaths in USA), being the second highest cause of dead in the country later form tobacco smoking related deaths~\cite{Mokdad}.

\section{Tentative Solution}
The good news is that people can prevent further damage by consistently exercise and by having a  balanced diet ,but changing exercise and nutritional habits is not easy.

The "Zombie Apocalypse Prep" application can help to reduce this problem, by providing a platform that will help exercise enthusiasts (and not so enthusiastic people) to accomplish their health goals and have fun while doing so. 

But what does this program do that other fitness apps did not? Well, by making the experience of exercise and nutrition a task oriented, dynamic  game, this app can have a higher likelihood to keep the user engaged in finishing their proposed goals. 

The app is built around the idea of surviving a zombie apocalypse, each task or goal completed adds to a cumulative score that will help the user to "survive" the simulated apocalypse. 
The app also contains a weight tracker, a diet planner (with nutrition plans integrated), a challenge mode, an exercise partner finder, trivia knowledge about science and engineering, and more.


\section{Challenges}
Some of the challenges that we may find, is how to keep users interested in the app while they see results from its use; how to give a fair point balance based in goals and challenges accomplished; how to make the app versatile and intuitive to use, among others.


\section{Resources}
We are planning to build this app by either making it a mobile application using Android studio, java, C\# and python languages or implementing it as a web application written in JavaScript, php and  other web compatible language. 

\section{Conclusion}
this application offers a huge spectrum of potential ramifications, that make it not just a fitness app, but a multi-use platform, that can be implemented for educating people about health and nutrition, helping people to live a healthier life, promote networking and relationship building and all this while you are having fun. 


\bibliography{myref}
\bibliographystyle{plain}

\end{document}